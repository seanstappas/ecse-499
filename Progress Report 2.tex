\documentclass[]{article}
\usepackage{fullpage}
\usepackage{graphicx}
\usepackage[lowtilde]{url}
\usepackage{hyperref}
\hypersetup{
	colorlinks=true,
	linkcolor=black,
	filecolor=magenta,      
	urlcolor=blue,
	citecolor=black
}
\usepackage{indentfirst}	% auto-indent
\usepackage[super]{nth}
\usepackage{cleveref}

%opening
\title{\textbf{Honours Thesis II \\ Progress Report 2}}
\author{Sean Stappas \\ 260639512}
\date{October \nth{11}, 2017}

\begin{document}
	\maketitle
	
	\section{Project Title \& Description}
	
	The project is \emph{Prometheus AI: Phase 2}. The Prometheus AI is a software model of the human brain with the goal of controlling and coordinating multiple robots in different environments. As an Honours Thesis, this project is to be done alone. As such, all sections on group work (list of group members and group work report) have been omitted.
	
	\section{Project Supervisor}
	
	The project supervisor and advisor is Joseph Vybihal, professor in computer science at McGill University. Contact information can be found on his website: \url{http://www.cs.mcgill.ca/~jvybihal/}
	
	\section{Meetings}
	% Provide a list of group meeting with dates and a very brief summary of topic of each meeting. Provide a list of dates when you met with your project supervisor(s).
	
	A meeting occurred on September \nth{29} with Prof. Vybihal, where the overall plan for the project this semester was discussed. Notably, future milestones I outlined in Prof. Vybihal's spreadsheet were addressed. These include following up with Isaac and Si Yi, restructuring the code and working on either new features or other layers of Prometheus. These are discussed in \Cref{sec:progress,sec:plans}.
	
	There was also a meeting with Si Yi, one of the students who worked on Prometheus over the summer, on October \nth{6}. Here we discussed how he could improve the quality of the code he wrote and addressed some issues with his implementation of lambda searching.
	
	Finally, there was a meeting with Prof. Vybihal on October \nth{6} as well. We discussed the recent progress (following up with Si Yi and Isaac) and the logical next steps.
	
	\section{Project Readings} \label{sec:readings}
	% Provide a bibliography of any material you have read since the beginning of the semester and any material you have identified for future reading in relation to your project.
	
	The readings for the past two weeks are the same as for the first progress report. Most of the recent work was in reading and writing code, as will be described in the next section.
	
	\section{Recent Progress} \label{sec:progress}
	% Summarize your progress since your last report. Compare this your original plans, and identify difficulties that hindered your progress and the steps you are taking to overcome them. (1/2 page maximum).
	
	In the past two weeks, I worked on getting Isaac and Si Yi to implement some needed clean-up on their code changes. As mentioned in the previous progress report, these two students worked on my ES and KNN code over the summer. The changes I suggested were mostly to encourage better documentation through Javadoc and to refactor very long methods into smaller ones that can be more easily digested by a human. Both these changes allow the code to be more easily read, which reduces the burden on potential new developers coming into the project. Both Si Yi and Isaac implemented most of the changes I suggested last week.
	
	During the meeting with Si Yi, some further issues were identified with his implementation of lambda searching. The main problem was the fact that, to compute the belief associated with nodes found during lambda search, node probabilities were multiplied as if they were global, but they were in fact conditional. In effect, Bayes' rule was being neglected. For now, however, focus was put on cleaning up the code. Si Yi will address this issue in the future.
	
	\section{Future Plans} \label{sec:plans}
	% Summarize your plans for future work for the period until the next report is due. Provide an expected timeline for your progress (1/2 page maximum).
		
	The plan until next progress report is to restructure the current code base to make it more unit-testable. This will be done with Google Guice, Mockito, and a testing library like TestNG or JUnit. This work may be challenging and require a lot of refactoring because a structure with dependency injection was not what was in mind when most of the code was written.
	
	Once this is done, some standard test cases for the ES and KNN will be discussed with Prof. Vybihal. After that, I will either work on implementing new features like learning, or work on building prototypes of the NN and META layers of Prometheus, similar to the initial work done on the ES and KNN. Even though work was started on a neural network by other students, it would not be possible to merge that code with the work done on the ES and KNN. This is why it would be more worthwhile to simply build skeletons of the other layers which can easily interact with the layers I constructed.
	
	%\renewcommand\refname{}
	%\bibliographystyle{unsrt}
	%\bibliography{readings}{}
	
\end{document}
