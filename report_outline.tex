\documentclass[11pt]{article}
\usepackage[utf8]{inputenc}
\usepackage{fullpage}
\usepackage{indentfirst}
\usepackage[per-mode=symbol]{siunitx}
\usepackage{listings}
\usepackage{graphicx}
\usepackage{color}
\usepackage{amsmath}
\usepackage{array}
\usepackage[hidelinks]{hyperref}
\usepackage[format=plain,font=it]{caption}
\usepackage{subcaption}
\usepackage{standalone}
\usepackage[nottoc]{tocbibind}
\usepackage{scrextend}
\usepackage[margin=1in]{geometry}
\usepackage{lstautodedent}
\usepackage[super]{nth}
\usepackage{todonotes}
\usepackage{titling}


\definecolor{dkgreen}{rgb}{0,0.6,0}
\definecolor{gray}{rgb}{0.5,0.5,0.5}
\definecolor{mauve}{rgb}{0.58,0,0.82}

\lstset{frame=tb,
  language=Java,
  aboveskip=3mm,
  belowskip=3mm,
  showstringspaces=false,
  columns=flexible,
  basicstyle={\small\ttfamily},
  keywordstyle=\color{blue},
  commentstyle=\color{dkgreen},
  stringstyle=\color{mauve},
  breaklines=true,
  breakatwhitespace=true,
  tabsize=3,
  basicstyle=\scriptsize\tt,
  autodedent,
  numbers=left,
  numberstyle=\tiny\color{gray},
  xleftmargin=2.5em,
  frame=single,
  framexleftmargin=2em
}
% PC
%\def \espath {"C:/Users/Sean/IdeaProjects/Prometheus/src/es/ExpertSystem.java"}
%\def \knnpath {"C:/Users/Sean/IdeaProjects/Prometheus/src/knn/KnowledgeNodeNetwork.java"}

% OSX
\def \espath {"/Users/seanstappas1/GitHub/prometheus-ai/src/main/java/es/api/ExpertSystem.java"}
\def \knnpath {"/Users/seanstappas1/GitHub/prometheus-ai/src/main/java/knn/api/KnowledgeNodeNetwork.java"}

\def\equationautorefname~#1\null{%
  Equation~(#1)\null
}

\def\sectionautorefname~#1\null{%
  Section~#1\null
}

\def\subsectionautorefname~#1\null{%
  Subsection~#1\null
}

\def\arraystretch{1.3}%  1 is the default, change whatever you need

% Custom commands
\newcommand{\ar}[1]{\autoref{#1}}
\newcommand\numberthis{\addtocounter{equation}{1}\tag{\theequation}}
\newcolumntype{P}[1]{>{\centering\arraybackslash}p{#1}}
\newcommand{\code}[1]{\texttt{#1}}
\newcommand{\specialcell}[2][c]{%
	\begin{tabular}[#1]{@{}c@{}}#2\end{tabular}}

\setlength{\droptitle}{-7em}
\title
{
	\Large\textbf{Prometheus Final Report Outline} \\ 
	\large ECSE 499: Honours Thesis II
}
\author % (optional, for multiple authors)
{
	Sean Stappas \\
	\small Supervised by: Prof. Joseph Vybihal
}
\date{November \nth{10}, 2017}

\begin{document}
	
\sloppy

\twocolumn
\maketitle

\section*{Abstract}
\begin{itemize}
	\item Summarize the entire thesis.
\end{itemize}

\section*{Acknowledgments}
\begin{itemize}
	\item Mention the contributions of other undergraduate students to the project.
\end{itemize}

\section{Introduction}
\begin{itemize}
	\item Introduce the sections to be covered.
\end{itemize}

\section{Background}
\begin{itemize}
	\item Describe the Prometheus AI layers, with an emphasis on the KNN and ES.
	\item Address the comments provided on last semester's report.
	\item Expand on some of the new theory provided in the Knowledge Nodes doc.
\end{itemize}

\section{Problem}
\begin{itemize}
	\item Describe the project requirements (creating and supervising the creation of Prometheus, with an emphasis on the KNN and ES).
\end{itemize}

\section{Design}
\begin{itemize}
	\item Describe the major design criteria (efficiency, OOP, readability, documentation, testability).
	\item Expand on the OOP criterion by describing fundamental (SOLID) principles like the Liskov substitution principle, dependency inversion, etc.
	\item Expand on the testability criterion by showing the necessity of unit testing and behavior-driven development (BDD).
\end{itemize}

\section{Implementation}
\begin{itemize}
	\item Describe how each of the previous design criteria were implemented in Prometheus.
	\item Describe the overall code structure (packages and Guice modules).
	\item Explain the use of the major libraries used (Google Guice, Mockito, TestNG and Apache Commons Lang 3).
\end{itemize}

\section{Results \& Tests}
\begin{itemize}
	\item Describe the unit tests.
	\item Describe the integration tests.
\end{itemize}

\section{Impact on Society and the Environment}
\begin{itemize}
	\item Describe possible impact areas (safety, risk, environmental and societal benefits).
\end{itemize}

\section{Conclusion}
\begin{itemize}
	\item Summarize accomplishments of the thesis.
	\item Discuss some insights (people/time management and technical debt).
	\item Discuss possible future extensions to the project.
\end{itemize}

\end{document}